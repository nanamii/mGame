\clearpage
\chapter{Lastenheft}

\section{Zielbestimmung} %welche Ziele sollen mit dem Software-Produkt erreicht werden.

\section{Produkteinsatz} %Anwendungsbereiche und Stakeholders werden genannt

\section{Produktfunktionen} %Hauptfunktionen werden beschrieben, Stakeholdergruppen
%zugeordnet und in 10er-Schritten durchnummeriert (LF nn).

$\backslash$LF10$\backslash$ \hspace{15 mm}Spieler anlegen \\
$\backslash$LF11$\backslash$ \hspace{15 mm}Spieler laden\\ % prüfen, ob Spieler bereits vorhanden
$\backslash$LF12$\backslash$ \hspace{15 mm}Spielerdaten ändern\\ \\  % löschen, ändern usw.

\noindent$\backslash$LF20$\backslash$ \hspace{15 mm}Spielmodus wählbar - 1 Spieler oder 2 Spieler\\
\noindent$\backslash$LF30$\backslash$ \hspace{15 mm}Spielfeldgröße wählbar\\
\noindent$\backslash$LF40$\backslash$ \hspace{15 mm}Thema wählbar\\

%\noindent$\backslash$LF20$\backslash$ \hspace{15 mm}Spielkarten verwalten\\
%\noindent$\backslash$LF30$\backslash$ \hspace{15 mm}Spielfeld erzeugen\\
%\noindent$\backslash$LF40$\backslash$ \hspace{15 mm}Spielzüge verwalten\\

\noindent$\backslash$LF50$\backslash$ \hspace{15 mm}Punktestand verwalten\\

\noindent $\backslash$LF60$\backslash$ \hspace{15 mm}Highscore erstellen\\
$\backslash$LF61$\backslash$ \hspace{15 mm}Highscore anzeigen\\
$\backslash$LF62$\backslash$ \hspace{15 mm}Urkunde drucken\\

\noindent$\backslash$LF70$\backslash$ \hspace{15 mm} Vokabeltraining\\ \\

\noindent$\backslash$LF71$\backslash$ \hspace{15 mm}Audiodaten abspielen \\ 


\section{Produktdaten} %permanent gespeicherte Hauptdaten werden festgelegt und in
%10er-Schritten durchnummeriert (LD nn).
$\backslash$LD10$\backslash$ \hspace{15 mm}Spielerdaten \\ \\
\noindent $\backslash$LD20$\backslash$ \hspace{15 mm}Highscoredaten \\ \\

\section{Produktleistungen} % besondere Anforderungen an Hauptfunktionen oder Haupt-
%daten (Ausführungszeit, Datenumfang, ... ) werden aufgezählt (LL nn).

\section{Qualitätsanforderungen} % allgemeine Eigenschaften wie gute Zuverlässigkeit,
%hervorragende Benutzbarkeit, normale Effizienz, ... werden festgelegt.

\section{Ergänzungen} % alles was nicht in obiges Schema passt und trotzdem wichtig ist.


