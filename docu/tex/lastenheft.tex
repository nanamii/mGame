\clearpage
\chapter{Lastenheft}

\section{Zielbestimmung} %welche Ziele sollen mit dem Software-Produkt erreicht werden.
Mit der Lernsoftware wird die Möglichkeit geschaffen, das pädagogisch wertvolle Prinzip des klassischen Memory-Spiel auf eine digitale Plattform zu übertragen. Zusätzlich zum Gedächtnistraining trägt die Vokabelfunktion zur Erweiterung des Wortschatzes bei. Die Führung einer Highscore ermöglicht den Vergleich der erreichten Punkte und steigert die Motivation der Benutzer, eine bessere Platzierung zu erreichen. Das erhöht wiederum den Lerneffekt. Neben dem Spielen an sich wird der Umgang mit dem Computer spielerisch erlernt. Vor Spielbeginn sind Attribute festzulegen und Meldungen der Software zu beachten. Das schult zusätzlich Logik. Der 2-Player-Modus unterstützt außerdem die Kommunikation mit anderen Spielern.


\section{Produkteinsatz} %Anwendungsbereiche und Stakeholders werden genannt
Die Lernsoftware kommt in den Kindertagesstätten der Diakonie Hochfranken zum Einsatz. Anwender der Software sind Besucher der Kindertagesstätte im Alter von fünf bis zehn Jahren.

\section{Produktfunktionen} %Hauptfunktionen werden beschrieben, Stakeholdergruppen
%zugeordnet und in 10er-Schritten durchnummeriert (LF nn).

$\backslash$LF10$\backslash$ \hspace{15 mm}Spieler neu erstellen \\
$\backslash$LF11$\backslash$ \hspace{15 mm}Spieler laden\\ % prüfen, ob Spieler bereits vorhanden
$\backslash$LF12$\backslash$ \hspace{15 mm}Spielerdaten ändern\\ \\  % löschen, ändern usw.

\noindent$\backslash$LF20$\backslash$ \hspace{15 mm}Anzahl der Spieler wählen\\
\noindent$\backslash$LF30$\backslash$ \hspace{15 mm}Thema wählen\\
\noindent$\backslash$LF40$\backslash$ \hspace{15 mm}Spielfeldgröße wählen\\

%\noindent$\backslash$LF20$\backslash$ \hspace{15 mm}Spielkarten verwalten\\
%\noindent$\backslash$LF30$\backslash$ \hspace{15 mm}Spielfeld erzeugen\\
%\noindent$\backslash$LF40$\backslash$ \hspace{15 mm}Spielzüge verwalten\\

\noindent$\backslash$LF50$\backslash$ \hspace{15 mm}Spiel starten\\

\noindent $\backslash$LF60$\backslash$ \hspace{15 mm}Highscore anzeigen\\
$\backslash$LF61$\backslash$ \hspace{15 mm}Urkunde drucken\\

\noindent$\backslash$LF70$\backslash$ \hspace{15 mm} Vokabeltraining\\ \\
\noindent$\backslash$LF80$\backslash$ \hspace{15 mm}Audiodaten abspielen \\ 


\section{Produktdaten} %permanent gespeicherte Hauptdaten werden festgelegt und in
%10er-Schritten durchnummeriert (LD nn).
$\backslash$LD10$\backslash$ \hspace{15 mm}Spielerdaten \\ \\
\noindent $\backslash$LD20$\backslash$ \hspace{15 mm}Highscoredaten \\ \\

\section{Produktleistungen} % besondere Anforderungen an Hauptfunktionen oder Haupt-
%daten (Ausführungszeit, Datenumfang, ... ) werden aufgezählt (LL nn).

\section{Qualitätsanforderungen} % allgemeine Eigenschaften wie gute Zuverlässigkeit,
%hervorragende Benutzbarkeit, normale Effizienz, ... werden festgelegt.
Funktionalität:	\hspace{15 mm}		gut\\
Zuverlässigkeit:\hspace{15 mm}		sehr gut\\
Benutzbarkeit:	\hspace{15 mm}		gut\\
Effizienz:	\hspace{15 mm}		normal\\
Änderbarkeit:	\hspace{15 mm}		normal\\
Portierbarkeit:	\hspace{15 mm}		sehr gut\\
Spassfaktor:	\hspace{15 mm}		sehr gut\\
\section{Ergänzungen} % alles was nicht in obiges Schema passt und trotzdem wichtig ist.
Die Umsetzung der Software erfolgt in der Programmiersprache Java. Da der Kunde Linux und Windows als Betriebssystem einsetzt stellt dies die notwendige Portierbarkeit sicher.


