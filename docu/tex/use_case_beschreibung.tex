\clearpage

\chapter{Use Case Beschreibungen}


\section{Spieler erstellen}
\begin{tabular}{|l|l|l|l|l|l|l|}
\hline
\textbf{Use Case:} & \multicolumn{ 6}{l|}{Spieler erstellen} \\ \hline
\textbf{Actors:} & \multicolumn{ 6}{l|}{Benutzer} \\ \hline
\textbf{Purpose:} & \multicolumn{ 6}{l|}{Benutzer erstellt durch Eingabe seines Namens einen Spieler} \\ \hline
\textbf{Entry Cond:} & \multicolumn{ 6}{c|}{} \\ \hline
\textbf{Overview:} & \multicolumn{ 6}{l|}{Für den Benutzer ist noch kein Spieler erstellt. Der Benutzer trägt seinen Namen ein 
und erstellt somit einen neuen Spieler.} \\ \hline
\textbf{Exit Cond:} & \multicolumn{ 6}{l|}{} \\ \hline
\textbf{Includes:} & \multicolumn{ 6}{l|}{} \\ \hline
\textbf{Special Req:} & \multicolumn{ 6}{l|}{} \\ \hline
\textbf{Category:} & \multicolumn{ 6}{l|}{} \\ \hline
\textbf{Cross Ref:} & \multicolumn{ 6}{l|}{auf /LF10/ aus Lastenheft} \\ \hline
\textbf{Ablauf:} & \multicolumn{ 3}{l|}{Actor Action:} & \multicolumn{ 3}{l|}{System Response:} \\ \hline
\multicolumn{ 1}{|c|}{} & \multicolumn{ 3}{l|}{1. Spieler erstellen wählen} & \multicolumn{ 3}{l|}{} \\ \cline{ 2- 7}
\multicolumn{ 1}{|l|}{} & \multicolumn{ 3}{l|}{2. Spielernamen eintragen} & \multicolumn{ 3}{l|}{} \\ \cline{ 2- 7}
\multicolumn{ 1}{|l|}{} & \multicolumn{ 3}{l|}{} & \multicolumn{ 3}{l|}{} \\ \cline{ 2- 7}
\multicolumn{ 1}{|l|}{} & \multicolumn{ 3}{l|}{} & \multicolumn{ 3}{l|}{} \\ \cline{ 2- 7}
\multicolumn{ 1}{|l|}{} & \multicolumn{ 3}{l|}{} & \multicolumn{ 3}{l|}{} \\ \cline{ 2- 7}
\multicolumn{ 1}{|l|}{} & \multicolumn{ 3}{l|}{} & \multicolumn{ 3}{l|}{} \\ \hline
\end{tabular}


\clearpage
\section{Spieler laden}

\clearpage
\section{Spielmodus wählen}

\clearpage
\section{Thema wählen}
